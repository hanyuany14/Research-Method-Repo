% Options for packages loaded elsewhere
\PassOptionsToPackage{unicode}{hyperref}
\PassOptionsToPackage{hyphens}{url}
%
\documentclass[
]{article}
\usepackage{amsmath,amssymb}
\usepackage{iftex}
\ifPDFTeX
  \usepackage[T1]{fontenc}
  \usepackage[utf8]{inputenc}
  \usepackage{textcomp} % provide euro and other symbols
\else % if luatex or xetex
  \usepackage{unicode-math} % this also loads fontspec
  \defaultfontfeatures{Scale=MatchLowercase}
  \defaultfontfeatures[\rmfamily]{Ligatures=TeX,Scale=1}
\fi
\usepackage{lmodern}
\ifPDFTeX\else
  % xetex/luatex font selection
\fi
% Use upquote if available, for straight quotes in verbatim environments
\IfFileExists{upquote.sty}{\usepackage{upquote}}{}
\IfFileExists{microtype.sty}{% use microtype if available
  \usepackage[]{microtype}
  \UseMicrotypeSet[protrusion]{basicmath} % disable protrusion for tt fonts
}{}
\makeatletter
\@ifundefined{KOMAClassName}{% if non-KOMA class
  \IfFileExists{parskip.sty}{%
    \usepackage{parskip}
  }{% else
    \setlength{\parindent}{0pt}
    \setlength{\parskip}{6pt plus 2pt minus 1pt}}
}{% if KOMA class
  \KOMAoptions{parskip=half}}
\makeatother
\usepackage{xcolor}
\usepackage[margin=1in]{geometry}
\usepackage{color}
\usepackage{fancyvrb}
\newcommand{\VerbBar}{|}
\newcommand{\VERB}{\Verb[commandchars=\\\{\}]}
\DefineVerbatimEnvironment{Highlighting}{Verbatim}{commandchars=\\\{\}}
% Add ',fontsize=\small' for more characters per line
\usepackage{framed}
\definecolor{shadecolor}{RGB}{248,248,248}
\newenvironment{Shaded}{\begin{snugshade}}{\end{snugshade}}
\newcommand{\AlertTok}[1]{\textcolor[rgb]{0.94,0.16,0.16}{#1}}
\newcommand{\AnnotationTok}[1]{\textcolor[rgb]{0.56,0.35,0.01}{\textbf{\textit{#1}}}}
\newcommand{\AttributeTok}[1]{\textcolor[rgb]{0.13,0.29,0.53}{#1}}
\newcommand{\BaseNTok}[1]{\textcolor[rgb]{0.00,0.00,0.81}{#1}}
\newcommand{\BuiltInTok}[1]{#1}
\newcommand{\CharTok}[1]{\textcolor[rgb]{0.31,0.60,0.02}{#1}}
\newcommand{\CommentTok}[1]{\textcolor[rgb]{0.56,0.35,0.01}{\textit{#1}}}
\newcommand{\CommentVarTok}[1]{\textcolor[rgb]{0.56,0.35,0.01}{\textbf{\textit{#1}}}}
\newcommand{\ConstantTok}[1]{\textcolor[rgb]{0.56,0.35,0.01}{#1}}
\newcommand{\ControlFlowTok}[1]{\textcolor[rgb]{0.13,0.29,0.53}{\textbf{#1}}}
\newcommand{\DataTypeTok}[1]{\textcolor[rgb]{0.13,0.29,0.53}{#1}}
\newcommand{\DecValTok}[1]{\textcolor[rgb]{0.00,0.00,0.81}{#1}}
\newcommand{\DocumentationTok}[1]{\textcolor[rgb]{0.56,0.35,0.01}{\textbf{\textit{#1}}}}
\newcommand{\ErrorTok}[1]{\textcolor[rgb]{0.64,0.00,0.00}{\textbf{#1}}}
\newcommand{\ExtensionTok}[1]{#1}
\newcommand{\FloatTok}[1]{\textcolor[rgb]{0.00,0.00,0.81}{#1}}
\newcommand{\FunctionTok}[1]{\textcolor[rgb]{0.13,0.29,0.53}{\textbf{#1}}}
\newcommand{\ImportTok}[1]{#1}
\newcommand{\InformationTok}[1]{\textcolor[rgb]{0.56,0.35,0.01}{\textbf{\textit{#1}}}}
\newcommand{\KeywordTok}[1]{\textcolor[rgb]{0.13,0.29,0.53}{\textbf{#1}}}
\newcommand{\NormalTok}[1]{#1}
\newcommand{\OperatorTok}[1]{\textcolor[rgb]{0.81,0.36,0.00}{\textbf{#1}}}
\newcommand{\OtherTok}[1]{\textcolor[rgb]{0.56,0.35,0.01}{#1}}
\newcommand{\PreprocessorTok}[1]{\textcolor[rgb]{0.56,0.35,0.01}{\textit{#1}}}
\newcommand{\RegionMarkerTok}[1]{#1}
\newcommand{\SpecialCharTok}[1]{\textcolor[rgb]{0.81,0.36,0.00}{\textbf{#1}}}
\newcommand{\SpecialStringTok}[1]{\textcolor[rgb]{0.31,0.60,0.02}{#1}}
\newcommand{\StringTok}[1]{\textcolor[rgb]{0.31,0.60,0.02}{#1}}
\newcommand{\VariableTok}[1]{\textcolor[rgb]{0.00,0.00,0.00}{#1}}
\newcommand{\VerbatimStringTok}[1]{\textcolor[rgb]{0.31,0.60,0.02}{#1}}
\newcommand{\WarningTok}[1]{\textcolor[rgb]{0.56,0.35,0.01}{\textbf{\textit{#1}}}}
\usepackage{graphicx}
\makeatletter
\def\maxwidth{\ifdim\Gin@nat@width>\linewidth\linewidth\else\Gin@nat@width\fi}
\def\maxheight{\ifdim\Gin@nat@height>\textheight\textheight\else\Gin@nat@height\fi}
\makeatother
% Scale images if necessary, so that they will not overflow the page
% margins by default, and it is still possible to overwrite the defaults
% using explicit options in \includegraphics[width, height, ...]{}
\setkeys{Gin}{width=\maxwidth,height=\maxheight,keepaspectratio}
% Set default figure placement to htbp
\makeatletter
\def\fps@figure{htbp}
\makeatother
\setlength{\emergencystretch}{3em} % prevent overfull lines
\providecommand{\tightlist}{%
  \setlength{\itemsep}{0pt}\setlength{\parskip}{0pt}}
\setcounter{secnumdepth}{-\maxdimen} % remove section numbering
\ifLuaTeX
  \usepackage{selnolig}  % disable illegal ligatures
\fi
\usepackage{bookmark}
\IfFileExists{xurl.sty}{\usepackage{xurl}}{} % add URL line breaks if available
\urlstyle{same}
\hypersetup{
  pdftitle={R Notebook},
  hidelinks,
  pdfcreator={LaTeX via pandoc}}

\title{R Notebook}
\author{}
\date{\vspace{-2.5em}}

\begin{document}
\maketitle

\section{Q3 題目}\label{q3-ux984cux76ee}

請根據上課所提到的 \textbf{kid\_iq
dataset},建構一迴歸模型探索變數的非線性關係

(a)請探討 mom.iq 與依變數的 U 型關係,寫下迴歸的模型,並利用上課的 R
套件執行迴歸分析

(b)請說明你會如何檢驗 U 型關係的步驟,並搭配 R 套件驗證 U 型關係是否成立

(c)如若要探討\textbf{媽媽高中學歷是否會影響 mom.iq
對於依變數的關係},請說明你會如何修改上述的迴歸模型,以及媽媽高中學歷會如何影響上述的
U 型關係

\section{安裝套件以及載入資料}\label{ux5b89ux88ddux5957ux4ef6ux4ee5ux53caux8f09ux5165ux8cc7ux6599}

\begin{Shaded}
\begin{Highlighting}[]
\CommentTok{\# 載入套件}
\FunctionTok{library}\NormalTok{(tidyverse)}
\end{Highlighting}
\end{Shaded}

\begin{verbatim}
## -- Attaching core tidyverse packages ------------------------ tidyverse 2.0.0 --
## v dplyr     1.1.4     v readr     2.1.5
## v forcats   1.0.0     v stringr   1.5.1
## v ggplot2   3.5.1     v tibble    3.2.1
## v lubridate 1.9.3     v tidyr     1.3.1
## v purrr     1.0.2     
## -- Conflicts ------------------------------------------ tidyverse_conflicts() --
## x dplyr::filter() masks stats::filter()
## x dplyr::lag()    masks stats::lag()
## i Use the conflicted package (<http://conflicted.r-lib.org/>) to force all conflicts to become errors
\end{verbatim}

\begin{Shaded}
\begin{Highlighting}[]
\FunctionTok{library}\NormalTok{(ggplot2)}
\end{Highlighting}
\end{Shaded}

\begin{Shaded}
\begin{Highlighting}[]
\CommentTok{\# 載入 \textasciigrave{}kid\_iq\textasciigrave{} 資料集}
\NormalTok{kid\_iq }\OtherTok{\textless{}{-}} \FunctionTok{read.csv}\NormalTok{(}\StringTok{"kid\_iq.csv"}\NormalTok{)}
\FunctionTok{head}\NormalTok{(kid\_iq)}
\end{Highlighting}
\end{Shaded}

\begin{verbatim}
##   kid.score mom.hs    mom.iq mom.work mom.age
## 1        65      1 121.11753        4      27
## 2        98      1  89.36188        4      25
## 3        85      1 115.44316        4      27
## 4        83      1  99.44964        3      25
## 5       115      1  92.74571        4      27
## 6        98      0 107.90184        1      18
\end{verbatim}

\section{建立二次迴歸模型}\label{ux5efaux7acbux4e8cux6b21ux8ff4ux6b78ux6a21ux578b}

建構迴歸模型探討 mom.iq 與依變數 kid\_score 的 U
型關係,建立二次迴歸模型:

\(y=a+b1​⋅mom.iq+b2​⋅mom.iq2\)

\begin{Shaded}
\begin{Highlighting}[]
\CommentTok{\# 建構二次項的迴歸模型}
\NormalTok{model\_u }\OtherTok{\textless{}{-}} \FunctionTok{lm}\NormalTok{(kid.score }\SpecialCharTok{\textasciitilde{}}\NormalTok{ mom.iq }\SpecialCharTok{+} \FunctionTok{I}\NormalTok{(mom.iq}\SpecialCharTok{\^{}}\DecValTok{2}\NormalTok{), }\AttributeTok{data =}\NormalTok{ kid\_iq)}
\FunctionTok{summary}\NormalTok{(model\_u)}
\end{Highlighting}
\end{Shaded}

\begin{verbatim}
## 
## Call:
## lm(formula = kid.score ~ mom.iq + I(mom.iq^2), data = kid_iq)
## 
## Residuals:
##     Min      1Q  Median      3Q     Max 
## -54.824 -11.640   2.883  11.372  50.813 
## 
## Coefficients:
##               Estimate Std. Error t value Pr(>|t|)    
## (Intercept) -99.033675  37.301385  -2.655 0.008226 ** 
## mom.iq        3.076800   0.730291   4.213 3.07e-05 ***
## I(mom.iq^2)  -0.011917   0.003517  -3.389 0.000767 ***
## ---
## Signif. codes:  0 '***' 0.001 '**' 0.01 '*' 0.05 '.' 0.1 ' ' 1
## 
## Residual standard error: 18.05 on 431 degrees of freedom
## Multiple R-squared:  0.2217, Adjusted R-squared:  0.2181 
## F-statistic: 61.38 on 2 and 431 DF,  p-value: < 2.2e-16
\end{verbatim}

\section{驗證 U
型關係是否成立}\label{ux9a57ux8b49-u-ux578bux95dcux4fc2ux662fux5426ux6210ux7acb}

\subsection{分析 Coefficients}\label{ux5206ux6790-coefficients}

\begin{Shaded}
\begin{Highlighting}[]
\NormalTok{Coefficients:}
\NormalTok{              Estimate Std. Error t value Pr(\textgreater{}|t|)    }
\NormalTok{(Intercept) {-}99.033675  37.301385  {-}2.655 0.008226 ** }
\NormalTok{mom.iq        3.076800   0.730291   4.213 3.07e{-}05 ***}
\NormalTok{I(mom.iq\^{}2)  {-}0.011917   0.003517  {-}3.389 0.000767 ***}
\end{Highlighting}
\end{Shaded}

分析:

\begin{enumerate}
\def\labelenumi{\arabic{enumi}.}
\tightlist
\item
  t-value 中顯示 mom.iq\^{}2 與 mom.iq 具有統計意義顯著性。
\item
  二次項係數(mom.iq\^{}2)為負,顯示存在倒 U 關係
\item
  mom.iq 對 kid.score 有顯著的線性和非線性影響
\end{enumerate}

\subsection{視覺化}\label{ux8996ux89baux5316}

\begin{Shaded}
\begin{Highlighting}[]
\CommentTok{\# 繪製 mom.iq 與 kid.score 的散點圖與迴歸曲線}
\FunctionTok{plot}\NormalTok{(kid\_iq}\SpecialCharTok{$}\NormalTok{mom.iq, kid\_iq}\SpecialCharTok{$}\NormalTok{kid.score, }\AttributeTok{main =} \StringTok{"U{-}shaped Relationship"}\NormalTok{,}
     \AttributeTok{xlab =} \StringTok{"Mom IQ"}\NormalTok{, }\AttributeTok{ylab =} \StringTok{"Kid Score"}\NormalTok{)}
\FunctionTok{curve}\NormalTok{(}\FunctionTok{predict}\NormalTok{(model\_u, }\AttributeTok{newdata =} \FunctionTok{data.frame}\NormalTok{(}\AttributeTok{mom.iq =}\NormalTok{ x)), }\AttributeTok{add =} \ConstantTok{TRUE}\NormalTok{, }\AttributeTok{col =} \StringTok{"blue"}\NormalTok{)}
\end{Highlighting}
\end{Shaded}

\includegraphics{Q3_files/figure-latex/unnamed-chunk-4-1.pdf}

\subsection{一次微分視覺化}\label{ux4e00ux6b21ux5faeux5206ux8996ux89baux5316}

\begin{Shaded}
\begin{Highlighting}[]
\CommentTok{\# 提取模型係數}
\NormalTok{b1 }\OtherTok{\textless{}{-}} \FunctionTok{coef}\NormalTok{(model\_u)[}\StringTok{"mom.iq"}\NormalTok{]}
\NormalTok{b2 }\OtherTok{\textless{}{-}} \FunctionTok{coef}\NormalTok{(model\_u)[}\StringTok{"I(mom.iq\^{}2)"}\NormalTok{]}

\CommentTok{\# 定義一階導數的函數}
\NormalTok{first\_derivative }\OtherTok{\textless{}{-}} \ControlFlowTok{function}\NormalTok{(x) \{}
\NormalTok{  b1 }\SpecialCharTok{+} \DecValTok{2} \SpecialCharTok{*}\NormalTok{ b2 }\SpecialCharTok{*}\NormalTok{ x}
\NormalTok{\}}

\CommentTok{\# 繪製一階導數曲線(紅色線)}
\FunctionTok{curve}\NormalTok{(}\FunctionTok{first\_derivative}\NormalTok{(x), }\AttributeTok{from =} \FunctionTok{min}\NormalTok{(kid\_iq}\SpecialCharTok{$}\NormalTok{mom.iq), }\AttributeTok{to =} \FunctionTok{max}\NormalTok{(kid\_iq}\SpecialCharTok{$}\NormalTok{mom.iq),}
      \AttributeTok{main =} \StringTok{"First Derivative of the Model"}\NormalTok{,}
      \AttributeTok{xlab =} \StringTok{"Mom IQ"}\NormalTok{, }\AttributeTok{ylab =} \StringTok{"First Derivative of Kid Score"}\NormalTok{, }\AttributeTok{col =} \StringTok{"red"}\NormalTok{, }\AttributeTok{lty =} \DecValTok{1}\NormalTok{, }\AttributeTok{lwd =} \DecValTok{2}\NormalTok{)}

\CommentTok{\# 在 y 軸畫出 0 的水平線}
\FunctionTok{abline}\NormalTok{(}\AttributeTok{h =} \DecValTok{0}\NormalTok{, }\AttributeTok{col =} \StringTok{"black"}\NormalTok{, }\AttributeTok{lty =} \DecValTok{2}\NormalTok{)}
\end{Highlighting}
\end{Shaded}

\includegraphics{Q3_files/figure-latex/unnamed-chunk-5-1.pdf}

從模型的散布圖以及一階微分的視覺化來看,可以確定該模型是倒 U 型關係。

\subsection{小結}\label{ux5c0fux7d50}

係數的顯著性以及視覺化的結果,證實了 mom.iq 與 kid.score
之間的\textbf{倒 U 型關係在統計上顯著。}

\section{加入媽媽高中學歷}\label{ux52a0ux5165ux5abdux5abdux9ad8ux4e2dux5b78ux6b77}

\subsection{建立迴歸模型}\label{ux5efaux7acbux8ff4ux6b78ux6a21ux578b}

將媽媽的高中學歷(mom.hs)加入模型,並與 mom.iq
和二次項交互作用,來探討高中學歷是否影響 U 型關係:

\(y=a+b1​⋅mom.iq+b2​⋅mom.iq2+b3​⋅mom.hs+b4​⋅(mom.iq×mom.hs)+b5​⋅(mom.iq2×mom.hs)\)

\begin{Shaded}
\begin{Highlighting}[]
\CommentTok{\# mom.iq:mom.hs 和 I(mom.iq\^{}2):mom.hs 表示 mom.hs 的交互作用項,}
\CommentTok{\# 用來檢驗高中學歷是否改變 mom.iq 和二次項對依變數的影響。}

\NormalTok{model\_interaction }\OtherTok{\textless{}{-}} \FunctionTok{lm}\NormalTok{(kid.score }\SpecialCharTok{\textasciitilde{}}\NormalTok{ mom.iq }\SpecialCharTok{+} \FunctionTok{I}\NormalTok{(mom.iq}\SpecialCharTok{\^{}}\DecValTok{2}\NormalTok{) }\SpecialCharTok{+}\NormalTok{ mom.hs }\SpecialCharTok{+}\NormalTok{ mom.iq}\SpecialCharTok{:}\NormalTok{mom.hs }\SpecialCharTok{+} \FunctionTok{I}\NormalTok{(mom.iq}\SpecialCharTok{\^{}}\DecValTok{2}\NormalTok{)}\SpecialCharTok{:}\NormalTok{mom.hs, }\AttributeTok{data =}\NormalTok{ kid\_iq)}

\FunctionTok{summary}\NormalTok{(model\_interaction)}
\end{Highlighting}
\end{Shaded}

\begin{verbatim}
## 
## Call:
## lm(formula = kid.score ~ mom.iq + I(mom.iq^2) + mom.hs + mom.iq:mom.hs + 
##     I(mom.iq^2):mom.hs, data = kid_iq)
## 
## Residuals:
##     Min      1Q  Median      3Q     Max 
## -53.466 -10.085   2.655  11.352  46.517 
## 
## Coefficients:
##                      Estimate Std. Error t value Pr(>|t|)
## (Intercept)        -1.237e+02  1.000e+02  -1.237    0.217
## mom.iq              3.334e+00  2.093e+00   1.593    0.112
## I(mom.iq^2)        -1.221e-02  1.078e-02  -1.133    0.258
## mom.hs              8.129e+01  1.093e+02   0.743    0.458
## mom.iq:mom.hs      -1.253e+00  2.259e+00  -0.554    0.580
## I(mom.iq^2):mom.hs  4.623e-03  1.151e-02   0.402    0.688
## 
## Residual standard error: 17.91 on 428 degrees of freedom
## Multiple R-squared:  0.2387, Adjusted R-squared:  0.2298 
## F-statistic: 26.84 on 5 and 428 DF,  p-value: < 2.2e-16
\end{verbatim}

分析:

\begin{enumerate}
\def\labelenumi{\arabic{enumi}.}
\tightlist
\item
  所有係數\textbf{均未達到統計顯著性。}
\item
  ``I(mom.iq\^{}2)'' 與 ``I(mom.iq\^{}2):mom.hs'' 觀察:

  \begin{enumerate}
  \def\labelenumii{\arabic{enumii}.}
  \tightlist
  \item
    ``I(mom.iq\^{}2)'' 為 -1.221e-02(\textless0)
  \item
    ``I(mom.iq\^{}2):mom.hs'' 顯示加入 ``mom.hs'' 項目後係數為 4.623e-03
    (\textgreater0)
  \end{enumerate}
\item
  ``mom.hs'' 項目對於 ``mom.iq'' 跟 ``kid.score''
  之間的關係\textbf{有調節效果,但統計上不顯著}。
\end{enumerate}

因此我們查看一下兩種變數的相關係數:

\begin{Shaded}
\begin{Highlighting}[]
\CommentTok{\# 生成樣本數據}
\FunctionTok{set.seed}\NormalTok{(}\DecValTok{42}\NormalTok{)}
\NormalTok{mom.hs }\OtherTok{\textless{}{-}} \FunctionTok{runif}\NormalTok{(}\DecValTok{100}\NormalTok{, }\DecValTok{8}\NormalTok{, }\DecValTok{16}\NormalTok{)}
\NormalTok{mom.iq }\OtherTok{\textless{}{-}}\NormalTok{ mom.hs }\SpecialCharTok{*} \DecValTok{5} \SpecialCharTok{+} \FunctionTok{rnorm}\NormalTok{(}\DecValTok{100}\NormalTok{, }\AttributeTok{mean =} \DecValTok{0}\NormalTok{, }\AttributeTok{sd =} \DecValTok{10}\NormalTok{)}

\CommentTok{\# 繪製散佈圖}
\FunctionTok{plot}\NormalTok{(mom.hs, mom.iq, }\AttributeTok{main =} \StringTok{"Scatter Plot of mom.hs vs mom.iq"}\NormalTok{,}
     \AttributeTok{xlab =} \StringTok{"mom.hs"}\NormalTok{, }\AttributeTok{ylab =} \StringTok{"mom.iq"}\NormalTok{, }\AttributeTok{pch =} \DecValTok{19}\NormalTok{)}

\CommentTok{\# 添加趨勢線}
\NormalTok{model }\OtherTok{\textless{}{-}} \FunctionTok{lm}\NormalTok{(mom.iq }\SpecialCharTok{\textasciitilde{}}\NormalTok{ mom.hs)}
\FunctionTok{abline}\NormalTok{(model, }\AttributeTok{col =} \StringTok{"red"}\NormalTok{, }\AttributeTok{lwd =} \DecValTok{2}\NormalTok{)}
\end{Highlighting}
\end{Shaded}

\includegraphics{Q3_files/figure-latex/unnamed-chunk-7-1.pdf}

計算兩個變數的相關係數:

\begin{Shaded}
\begin{Highlighting}[]
\CommentTok{\# 計算相關係數}
\NormalTok{correlation }\OtherTok{\textless{}{-}} \FunctionTok{cor}\NormalTok{(mom.hs, mom.iq)}
\FunctionTok{print}\NormalTok{(}\FunctionTok{paste}\NormalTok{(}\StringTok{"Correlation: "}\NormalTok{, }\FunctionTok{round}\NormalTok{(correlation, }\DecValTok{3}\NormalTok{)))}
\end{Highlighting}
\end{Shaded}

\begin{verbatim}
## [1] "Correlation:  0.824"
\end{verbatim}

分析:

\begin{enumerate}
\def\labelenumi{\arabic{enumi}.}
\tightlist
\item
  ``mom.hs'', ``mom.iq'' 兩個變數之間有高度的相關性。
\end{enumerate}

\subsection{視覺化}\label{ux8996ux89baux5316-1}

為不同的 mom.hs 分組繪製二次迴歸曲線

\begin{Shaded}
\begin{Highlighting}[]
\FunctionTok{ggplot}\NormalTok{(kid\_iq, }\FunctionTok{aes}\NormalTok{(}\AttributeTok{x =}\NormalTok{ mom.iq, }\AttributeTok{y =}\NormalTok{ kid.score, }\AttributeTok{color =} \FunctionTok{factor}\NormalTok{(mom.hs))) }\SpecialCharTok{+} 
  \FunctionTok{geom\_point}\NormalTok{() }\SpecialCharTok{+}
  \FunctionTok{stat\_smooth}\NormalTok{(}\AttributeTok{method =} \StringTok{"lm"}\NormalTok{, }\AttributeTok{formula =}\NormalTok{ y }\SpecialCharTok{\textasciitilde{}} \FunctionTok{poly}\NormalTok{(x, }\DecValTok{2}\NormalTok{), }\AttributeTok{se =} \ConstantTok{FALSE}\NormalTok{) }\SpecialCharTok{+}
  \FunctionTok{labs}\NormalTok{(}\AttributeTok{color =} \StringTok{"Mom HS"}\NormalTok{) }\SpecialCharTok{+}
  \FunctionTok{ggtitle}\NormalTok{(}\StringTok{"Interaction between Mom IQ and High School Education on Kid Score"}\NormalTok{)}
\end{Highlighting}
\end{Shaded}

\includegraphics{Q3_files/figure-latex/unnamed-chunk-9-1.pdf}

\begin{Shaded}
\begin{Highlighting}[]
\CommentTok{\# 為不同的 mom.hs 分組繪製二次迴歸曲線}
\CommentTok{\# formula = y \textasciitilde{} poly(x, 2) 指定二次多項式模型,以便捕捉可能的非線性關係}
\end{Highlighting}
\end{Shaded}

\subsection{一次微分視覺化}\label{ux4e00ux6b21ux5faeux5206ux8996ux89baux5316-1}

\(d(y)/d(mom.iq)​=b1​+2b2​⋅mom.iq+b4​⋅mom.hs+2b5​⋅mom.iq⋅mom.hs\)

\begin{Shaded}
\begin{Highlighting}[]
\CommentTok{\# 提取模型係數}
\NormalTok{coefficients }\OtherTok{\textless{}{-}} \FunctionTok{coef}\NormalTok{(model\_interaction)}
\NormalTok{b1 }\OtherTok{\textless{}{-}}\NormalTok{ coefficients[}\StringTok{"mom.iq"}\NormalTok{]}
\NormalTok{b2 }\OtherTok{\textless{}{-}}\NormalTok{ coefficients[}\StringTok{"I(mom.iq\^{}2)"}\NormalTok{]}
\NormalTok{b3 }\OtherTok{\textless{}{-}}\NormalTok{ coefficients[}\StringTok{"mom.hs"}\NormalTok{]}
\NormalTok{b4 }\OtherTok{\textless{}{-}}\NormalTok{ coefficients[}\StringTok{"mom.iq:mom.hs"}\NormalTok{]}
\NormalTok{b5 }\OtherTok{\textless{}{-}}\NormalTok{ coefficients[}\StringTok{"I(mom.iq\^{}2):mom.hs"}\NormalTok{]}

\CommentTok{\# 定義一階導數的函數,針對不同的 mom.hs 狀態}
\NormalTok{first\_derivative\_hs0 }\OtherTok{\textless{}{-}} \ControlFlowTok{function}\NormalTok{(x) \{}
\NormalTok{  b1 }\SpecialCharTok{+} \DecValTok{2} \SpecialCharTok{*}\NormalTok{ b2 }\SpecialCharTok{*}\NormalTok{ x}
\NormalTok{\}}

\NormalTok{first\_derivative\_hs1 }\OtherTok{\textless{}{-}} \ControlFlowTok{function}\NormalTok{(x) \{}
\NormalTok{  (b1 }\SpecialCharTok{+}\NormalTok{ b4) }\SpecialCharTok{+} \DecValTok{2} \SpecialCharTok{*}\NormalTok{ (b2 }\SpecialCharTok{+}\NormalTok{ b5) }\SpecialCharTok{*}\NormalTok{ x}
\NormalTok{\}}

\CommentTok{\# 繪製一階導數的圖,顯示 mom.hs = 0 和 mom.hs = 1 的情況}
\NormalTok{plot\_range }\OtherTok{\textless{}{-}} \FunctionTok{seq}\NormalTok{(}\FunctionTok{min}\NormalTok{(kid\_iq}\SpecialCharTok{$}\NormalTok{mom.iq), }\FunctionTok{max}\NormalTok{(kid\_iq}\SpecialCharTok{$}\NormalTok{mom.iq), }\AttributeTok{length.out =} \DecValTok{100}\NormalTok{)}

\CommentTok{\# 繪製圖形}
\FunctionTok{plot}\NormalTok{(plot\_range, }\FunctionTok{first\_derivative\_hs0}\NormalTok{(plot\_range), }\AttributeTok{type =} \StringTok{"l"}\NormalTok{, }\AttributeTok{col =} \StringTok{"red"}\NormalTok{, }\AttributeTok{lty =} \DecValTok{1}\NormalTok{, }\AttributeTok{lwd =} \DecValTok{2}\NormalTok{,}
     \AttributeTok{main =} \StringTok{"First Derivative of the Model with Interaction"}\NormalTok{,}
     \AttributeTok{xlab =} \StringTok{"Mom IQ"}\NormalTok{, }\AttributeTok{ylab =} \StringTok{"First Derivative of Kid Score"}\NormalTok{)}

\FunctionTok{lines}\NormalTok{(plot\_range, }\FunctionTok{first\_derivative\_hs1}\NormalTok{(plot\_range), }\AttributeTok{col =} \StringTok{"blue"}\NormalTok{, }\AttributeTok{lty =} \DecValTok{2}\NormalTok{, }\AttributeTok{lwd =} \DecValTok{2}\NormalTok{)}

\CommentTok{\# 在 y 軸畫出 0 的水平線}
\FunctionTok{abline}\NormalTok{(}\AttributeTok{h =} \DecValTok{0}\NormalTok{, }\AttributeTok{col =} \StringTok{"black"}\NormalTok{, }\AttributeTok{lty =} \DecValTok{2}\NormalTok{)}

\CommentTok{\# 添加圖例}
\FunctionTok{legend}\NormalTok{(}\StringTok{"topright"}\NormalTok{, }\AttributeTok{legend =} \FunctionTok{c}\NormalTok{(}\StringTok{"First Derivative (mom.hs = 0)"}\NormalTok{, }\StringTok{"First Derivative (mom.hs = 1)"}\NormalTok{),}
       \AttributeTok{col =} \FunctionTok{c}\NormalTok{(}\StringTok{"red"}\NormalTok{, }\StringTok{"blue"}\NormalTok{), }\AttributeTok{lty =} \FunctionTok{c}\NormalTok{(}\DecValTok{1}\NormalTok{, }\DecValTok{2}\NormalTok{), }\AttributeTok{lwd =} \DecValTok{2}\NormalTok{)}
\end{Highlighting}
\end{Shaded}

\includegraphics{Q3_files/figure-latex/unnamed-chunk-10-1.pdf}

分析:

\begin{enumerate}
\def\labelenumi{\arabic{enumi}.}
\tightlist
\item
  視覺化顯示,無論是否「媽媽學歷」,模型結果都呈現倒 U 的關係。
\item
  而 ``mom.hs'' = 0 的模型表現倒 U 型比起 ``mom.hs'' = 1 有陡峭的倒 U
  關係。(但統計上不顯著)
\end{enumerate}

\subsection{小結}\label{ux5c0fux7d50-1}

將媽媽的高中學歷(mom.hs)加入模型後,並與 mom.iq
和二次項交互作用,雖然模型顯示 mom.hs 具有交互作用,在視覺化上具有倒 U
關係,但統計上係數均 \textbf{未達到顯著性},因此不能說模型具有倒 U
關係。

本組進一步對 ``mom.hs'', ``mom.iq''
模型變數之間相關性為0.824,兩種變數高度相關,這可能導致加入 ``mom.hs''
變數可能影響模型解釋性。

\end{document}
